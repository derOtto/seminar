\chapter{BSI}\label{ch:bsi}
Bereits im vorherigen Kapitel war oft vom BSI-Gesetz die Rede.
Das BSI beziehungsweise deren Gesetz spielt demnach eine zentrale Rolle im \acrshort{it-sig-2.0}.
Allerdings ging es im vorher meist um Pflichten und Aufgaben der Unternehmen (im Bezug zum BSI).
Im Folgenden jedoch geht es um das Innere des Bundesamts, deren Aufgaben und Befugnisse.

\section{Personal}\label{sec:personal}
Natürlich werden nicht nur in den KRITIS-Unternehmen selbst Mitarbeitende benötigt,
auch das BSI muss durch die Novelle an Personal aufstocken.
Im Gesetzentwurf der Bundesregierung wird im Erfüllungsaufwand mit 799~Planstellen und
jährlichen Personalkosten in Höhe von rund 74,24~Mio.~€ gerechnet.
Es ist damit einer der größten Aufwände dieses Gesetzes.

\section{Neue Aufgaben und Befugnisse}\label{sec:neue-aufgaben-und-befugnisse}
Auch das BSI hat künftig neue Aufgaben zu verfolgen.
Diese sind in §~3 BSI-Gesetz ergänzt worden und umfassen:
\begin{itemize}
    \item \enquote{Aufgaben und Befugnisse als nationale Behörde für Cybersicherheitszertifizierung (5a)}~\cite{neue-it-sig-2.0}
    \item \enquote{Erweiterte Beratung und Warnung staatlicher Stellen (12a und 14)}~\cite{neue-it-sig-2.0}
    \item \enquote{Verantwortungen für den Verbraucherschutz (14a)}~\cite{neue-it-sig-2.0}
    \item \enquote{Erweiterung als zentrale Stelle für KRITIS-Betreiber und UBI/UNBÖFI (17)}~\cite{neue-it-sig-2.0}
    \item \enquote{Empfehlungen für die Sicherheit von IAM-Verfahren und Stand der Technik für Sicherheit in IT-Produkten, nach bestehenden Normen (19 und 20)}~\cite{neue-it-sig-2.0}
\end{itemize}

Als zentrale Meldestelle für Sicherheit in der Informationstechnik,
\enquote{darf das BSI Informationen zu Schadprogrammen, Lücken und Angriffen sammeln, analysieren,
über noch einzurichtende Meldewege entgegennehmen und diese Informationen u.a. für Warnungen, Meldungen benutzen}~\cite{neue-it-sig-2.0}.
Das bedeutet, dass das Bundesamt selbst Schlüsse ziehen und im Bedarfsfall die Bevölkerung warnen kann und soll.

\subsection{Hackerbehörde}\label{ssec:hackerbehorde}
Zu kaum einem Thema des \acrshort{it-sig-2.0} gab es so viele Artikel, wie dass das BSI zur Hackerbehörde wird
(vgl.~\cite{bsi-wird-zur-hackerbehoerde,it-sig-2.0-bundestag-baut-bsi-zur-hackerbehoerde-aus,seehofer-bsi-hackerbehoerde,entwurf-hackerbehoerde}).
Ganz unbegründet sind diese jedoch auch nicht.
Gleich vier Paragrafen (7a – 7d) des BSI-Gesetzes wurden neu gefasst oder gar neu eingefügt.

\subsubsection{Untersuchung von IT-Produkten und Systemen}\label{sssec:untersuchung-von-it-produkten-und-systemen}
Der erste Paragraf (7a) befasst sich mit der \textsc{Untersuchung der Sicherheit in der Informationstechnik}.
Genauer darf das \acrshort{bsi} IT-Produkte und Systeme des Marktes untersuchen.
Auch kann es hierbei Auskünfte von Herstellern verlangen.
Die Ergebnisse dürfen dann, nach Einräumung einer Stellungsnahme des Herstellers, veröffentlicht oder weiter gegeben werden.
Kritisch zu sehen ist hier jedoch, dass es keine Pflicht zur Veröffentlichung gibt.
Erkenntnisse über Schwachstellen könnten demnach zum Beispiel zum eigenen Zweck zurückgehalten werden.
(vgl.~\cite{neue-it-sig-2.0})

\subsubsection{Portscans}\label{sssec:portscans}
\begin{figure}[H]\centering
    \includegraphics{chapters/images/shodan_exposure_germany}
    \caption{\href{https://exposure.shodan.io/\#/DE/}{Internet Exposure Dashboard – Germany} (vgl.~\cite{shodan-dashboard-exposure-germany})}
    \label{fig:shodan-exposure-germany}
\end{figure}
Kaum eine Suchmaschine ist so erschreckend wie \href{https://www.shodan.io/}{shodan.io}.
Während es für uns in Deutschland nur mit Auftrag und Genehmigung erlaubt ist,
über einen Portscan Systeme zu untersuchen, um so z.B.\ Sicherheitslücken zu finden,
so liefert die Suchmaschine gleich alle Daten fein sortiert und filterbar in einer aufbereiteten Form im Web.
Allerdings lassen sich damit nicht nur, wie so oft, ungeschützte Kameras finden,
auch Dashboards zu den einzelnen Ländern liefern so wichtige Daten.
In \cref{fig:shodan-exposure-germany} sieht man die gewonnenen Daten zu den Portscans deutscher IP-Adressen,
zum Zeitpunkt dieser Arbeit.
Besonders besorgniserregend ist in diesem Fall die Zahl der erreichbaren
Industriellen Steuerungs- und Automatisierungssysteme (\acrshort{ics}).
Das sind nämlich Systeme, die zum Messen, Steuern und Regeln von Abläufen in der Industrie,
meist in Branchen der KRITIS vorkommen.
(vgl.~\cite{bsi-ics})

§~7b (\textsc{Detektion von Sicherheitsrisiken für die Netz- und IT-Sicherheit und von Angriffsmethoden})
ermächtigt das \acrshort{bsi} nun zum aktiven Hacken.
Gemeint sind mit den Maßnahmen Portscans an öffentlich erreichbaren Schnittstellen.
So sollten solche, öffentlich erreichbare Systeme gefunden werden können.

Allerdings handelt es sich bei den Systemen, welche hierbei kontrolliert werden dürfen,
nur um einen Teilbereich.
Nur IP-Bereiche auf einer Weißen Listen von Systemen des Bundes, von KRITIS-Betreibern oder \acrshort{unboefi},
welche zudem stets anzupassen ist, dürfen dabei gescannt werden.
Sollte eine oder mehrere Lücken gefunden werden,
so soll unverzüglich die Verantwortlichen bzw.\ der Betreiber darüber in Kenntnis gesetz werden.
(vgl.~\cite{neue-it-sig-2.0})

\subsubsection{Störungsbeseitigung}\label{sssec:storungsbeseitigung}
Mit Paragraf 7c (\textsc{Anordnung des Bundesamtes gegenüber Diensteanbietern}) wurde dem BSI wohl eine der meist kritisierten Befugnisse erteilt.

Im Fall konkreter erheblicher Gefahren kann das Bundesamt \acrshort{tkg}-Anbietern mit mehr als 100.000 Kund*innen anordnen,
dass diese technische Befehle zur Bereinigung an betroffene Systeme verteilen.
Das könnten beispielsweise Sicherheitslücken in Routern von \acrshort{isp} sein,
welche dann von den \acrshort{isp} nach Anordnung automatisch gepacht werden müssten.
(vgl.~\cite{neue-it-sig-2.0})

Zusätzlich kann das Umleiten des Datenverkehrs an eine Anschlusskennung des \acrshort{bsi} angeordnet werden.
Die so bekommenen Daten dürfen dabei auch gespeichert und verarbeitet werden.
Wenngleich bis Mitte des Folgejahres die Gesamtzahl an angeordneten Datenumleitungen dem oder der \acrshort{bfdi} gemeldet werden muss,
so wird das aber auch sehr kritisch gesehen.
Das Problem ist dabei, dass so ein System schnell auch missbraucht werden könnte, wie zum Beispiel für Zensur.
(vgl.~\cite{neue-it-sig-2.0})

\subsubsection{Telemedien-Anbieter}\label{sssec:telemedien-anbieter}
§~7d regelt lediglich,
dass die Behörde Telemendien-Anbietern Anordnungen zur Ergreifung von Sicherheitsmaßnahmen erteilen darf.
(vgl.~\cite{neue-it-sig-2.0})

\subsection{Schutz der Bundesnetze}\label{ssec:schutz-der-bundesnetze}
Neben der Funktion als \enquote{Hackerbehörde},
werden mit dem Paragrafen 4a BSI-Gesetz dem \acrshort{bsi} die Rechte und Pflichten zum Schutz der Bundesnetze auferlegt.
Darunter fallen:
\begin{itemize}
    \item die Sicherheit der Kommunikationstechnik und Infrastrukturen des Bundes zu kontrollieren
    (Informationen und Dokumente dürfen dabei angefordert werden),
    \item der Zutritt zu Betriebsräumen der Kommunikationstechnik des Bundes,
    \item mit Zustimmung Dritter, deren Schnittstellen zur Kommunikationstechnik des Bundes zu kontrollieren.
\end{itemize}
Ausgenommen ist jedoch die \acrshort{ikt} des auswärtigen Dienstes und der Bundeswehr.
(vgl.~\cite{neue-it-sig-2.0})

Die Behörde darf, ermächtigt durch §~5a BSI-Gesetz, zum Erkennen von Störungen und Angriffen des Bundes,
behördeninterne Protokolldaten verarbeiten.
Demnach kann das \acrshort{bsi} das Monitoring Bundes-IT übernehmen.
Zusätzlich ist es berechtigt (durch §~8) verbindliche Mindeststandards für die Sicherheit der Bundes-IT festzulegen.
(vgl.~\cite{neue-it-sig-2.0})

\subsection{Zertifizierungen}\label{ssec:zertifizierungen}
Mit §~9a wird das \acrshort{bsi} die \textsc{Nationale Behörde für die Cybersicherheitszertifizierung} des Staats.
Damit kommt der Gesetzgeber der Umsetzung der Cybersicherheitsverodnung (ENISA) der EU nach.
Damit darf sie \enquote{Konformitätsbewertungsstellen eine Tätigkeitserlaubnis erteilten und Zertifikate überprüfen und wiederrufen}~\cite{neue-it-sig-2.0}.

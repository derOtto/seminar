\chapter{Sanktionen}\label{ch:sanktionen}
Nachdem wir uns nun mit den eigentlichen Inhalten, wie den Anforderungen der KRITS-Betreiber beschäftigt haben,
so soll dieses Kapitel kurz die Sanktionen ansprechen, welche bei Verstößen drohen.

In \cref{tab:ordnungswidrigkeiten} sieht man die Ordnungswidrigkeiten entsprechend mit ihren höchsten Bußgeldern.

\begin{sidewaystable}[htp]
    \centering
    \begin{tabular}{@{}p{0.6\textwidth}p{0.1\textwidth}p{0.15\textwidth}@{}}
        \toprule
        \textbf{Verstoß}                                                                                                                  & \textbf{BSI-Gesetz}                                                          & \begin{tabular}[c]{@{}l@{}}\textbf{Bußgeld}\\(bis zu, in Euro)\end{tabular} \\ \midrule
        KRITIS-Nachweise nicht richtig oder nicht vollständig erbringen                                                          & §8a (3)                                                             & 1 Mio., 10 Mio. als jur. Person / Organ               \\ \midrule
        KRITIS-Nachweise nicht oder nicht rechtzeitig erbringen                                                                  & §8a (3)                                                             & 1 Mio., 10 Mio. als jur. Person / Organ               \\ \midrule
        Systeme auf Verlangen des BSI nicht wiederherstellen oder keine Maßnahmen zur Gefahrenabwehr treffen                     & \begin{tabular}[c]{@{}l@{}}§5b (6)\\ §7c (1)\\ §8a (3)\end{tabular} & 2 Mio., 20 Mio. als jur. Person / Organ               \\ \midrule
        Fehlende Mitwirkung bei Störungsbeseitigung                                                                              & §8b (6)                                                             & 500 Tsd.                                                     \\ \midrule
        Fehlende, unvollständige oder verspätete Meldung von Störungen bei KRITIS, Digitalen Diensten oder UNBÖFI                & \begin{tabular}[c]{@{}l@{}}§8b (4)\\ §8c (3)\\ §8f (7)\end{tabular}   & 500 Tsd.                                                 \\ \midrule
        Fehlende Umsetzung von KRITIS Cyber Security Maßnahmen                                                                   & §8a (1)                                                             & 1 Mio., 10 Mio. als jur. Person / Organ           \\ \midrule
        Fehlende Maßnahmen von Anbietern digitaler Dienste (DSP)                                                                 & §8c (1)                                                             & 500 Tsd.                                                 \\ \midrule
        Fehlende, unvollständige Selbsterklärung zur IT-Sicherheit von UNBÖFI                                                    & §8f (1)                                                             & 500 Tsd.                                                 \\ \midrule
        Fehlende Registrierung als KRITIS, fehlende Kontaktstelle, keine Erreichbarkeit                                          & §8b (3)                                                             & 500 Tsd., 100 Tsd. bei fehlender Erreichbarkeit \\ \midrule
        Fehlende Registrierung als UNBÖFI                                                                                        & §8f (5)                                                             & 500 Tsd.                                                 \\ \midrule
        Hersteller-Informationen für Untersuchungen nicht herausgeben                                                            & §7a (2)                                                             & 100 Tsd.                                                 \\ \midrule
        Fehlende Auskünfte von Anbietern Digitaler Dienste                                                                       & §8c (4)                                                             & 500 Tsd.                                                 \\ \midrule
        Dem BSI Zutritt, Informationen oder Unterstützung bei der Überprüfung von Maßnahmen oder KRITIS-Registrierung verweigern & §8a (4) §8b (3a)                                                    & 100 Tsd.                                                 \\ \midrule
        Als Konformitäts­bewertungsstelle ohne Genehmigung tätig werden                                                          & §9a (2)                                                             & 500 Tsd.                                                 \\ \midrule
        Sicherheitskennzeichen ohne Freigabe verwenden                                                                           & §9c (4)                                                             & 500 Tsd.                                                 \\ \bottomrule
    \end{tabular}
    \caption{Ordnungswidrigkeiten und Bußgelder (vgl. ~\cite{bussgelder-in-kritis,neue-it-sig-2.0}) }
    \label{tab:ordnungswidrigkeiten}
\end{sidewaystable}

Wie man sieht ist sowohl die Liste an Ordnungswidrigkeiten, als auch die Höhe der Bußgelder deutlich gestiegen.
Vorher lag die Höchstgrenze bei 50 Tsd.\ Euro bzw.\ bei 100 Tsd.\ Euro bei Missachtung von Anordnungen.
Die teilweise maximale Höhe von 20 Millionen Euro deckt sich mit der, der \acrshort{dsgvo},
wenn man die Bemessung am jährlichen Umsatz außer Acht lässt.
Es zeigt damit die Wichtigkeit über die Einhaltung der Vorgaben, was auch voraussichtlich zum gewünschten Effekt führt,
wie man es in den vergangenen Jahren auch bei der Datenschutzverordnung gesehen hat.

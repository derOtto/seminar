\chapter{Umriss}\label{ch:umriss}
Zu Beginn soll mit diesem Kapitel zunächst die Historie des IT-Sicherheitsgesetzes 2.0 (hier und im Folgenden oft auch als \acrshort{it-sig-2.0} abgekürzt) beleuchtet werden.
Dabei wird auch Bezug zu anderen relevanten Gesetzen und \glspl{eu-richtlinie} genommen.

\section{Historie}\label{sec:historie}
%\newcommand{\MyIcon{\color{orange}\vrule width 5pt height5pt\relax}}

\setupchronograduation[event]{markdepth=2cm}

\begin{figure}[H]\centering
    \startchronology[startyear=2015, stopyear=2021, color=lightgray, height=7ex]
    \chronoevent[icon=\color{blue}\vrule width 5pt height 5pt,textstyle=\bf,datesstyle=\it,conversionmonth=false,markdepth=5cm]{08/2016}{EU NIS Richtlinie}
    \chronoevent[icon=\color{orange}\vrule width 5pt height 5pt,textstyle=\bf,datesstyle=\it,markdepth=3.5cm,conversionmonth=false]{06/2017}{KRITIS-Verordnung Korb 2}
    \chronoevent[icon=\color{orange}\vrule width 5pt height 5pt,textstyle=\bf,datesstyle=\it,markdepth=2cm,conversionmonth=false]{04/2016}{KRITIS-Verordnung Korb 1}
    \chronoevent[icon=\color{orange}\vrule width 5pt height 5pt,textstyle=\bf,datesstyle=\it,markdepth=1cm,conversionmonth=false]{07/2015}{IT-SiG 1.0}
    \chronoperiode[textstyle=\bf,datesstyle=\it,color=orange]{2019}{2021}{Entwürfe IT-SiG 2.0}
    \chronoevent[icon=\color{blue}\vrule width 5pt height 5pt,textstyle=\bf,datesstyle=\it,conversionmonth=false,markdepth=5cm]{2020}{Entwürfe EU NIS2 / RCE}
    \chronoevent[icon=\color{orange}\vrule width 5pt height 5pt,textstyle=\bf,datesstyle=\it,markdepth=1cm]{2021}{IT-Sicherheitsgesetz 2.0}
    \stopchronology
    \caption{Zeitlicher Ablauf wichtiger Gesetzte und Richtlinien im Bezug zum IT-Sicherheitsgesetz 2.0 (vgl.~\cite{briefing-it-sig-2.0})}
    \label{fig:historie-it-sig-2.0}
\end{figure}

In \cref{fig:historie-it-sig-2.0} sieht man den zeitlichen Ablauf, bis zur Verabschiedung des \acrshort{it-sig-2.0} im April / Mai 2021.

Bereits 2015 wurde mit dem \acrshort{it-sig-1.0} eine erste Grundlage geschaffen.
Damit wurden insbesondere Vorkehrungen im Meldewesen verschiedener Sektoren, wie Energie oder Telekommunikation getroffen,
aber auch Verschiedenes definiert.
Knapp ein Jahr später im April 2016 hat das \acrshort{bsi} mit der ersten Kritis-Verordnung das \acrshort{it-sig-1.0}
für den ersten Korb (Sektoren Energie, IT+TK, Ernährung, Wasser) umgesetzt.
Wieder gut ein Jahr später nämlich im Juni 2017 wurden durch die Kritis-Verordnung Korb 2 die Regelungen für die Sektoren
Gesundheit, Finanz- \& Versicherungswesen und Transport \& Verkehr spezifiziert.
(vgl.~\cite{kritis-gesetzgebung,umsetzung-it-sig-1.0-bsi})

Währenddessen trat im August 2016 die NIS-Richtlinie in Kraft.
Diese ist die erste Maßnahme für Cyber-Sicherheit auf Europa-Ebene.
Für Deutschland musste allerdings nicht mehr viel umgesetzt bzw. geändert werden,
da das Meiste bereits mit dem \acrshort{it-sig-1.0} geregelt war.
Komplett neu mussten letztlich Regelungen für \textsc{Anbieter Digitaler Dienste} geschaffen werden.
Gleichzeitig mit dem Gesetz zur Umsetzung der NIS-Richtlinie,
erhielt das \acrshort{bsi} mehr Befugnisse, sodass es künftig die Bundesländer besser unterstützen und beraten kann.
(vgl.~\cite{gesetz-zur-umsetzung-der-nis-richtlinie})

Bis das IT-Sicherheitsgesetz 2.0 letztlich in der beschlossenen Fassung im Mai 2021 beschlossen wurde,
gab es ab Mai 2019 etliche Versionen.
Die AG KRITIS listet in ihrem Artikel \url{https://ag.kritis.info/2021/05/19/it-sicherheitsgesetz-2-0-alle-verfuegbaren-versionen/} alle verfügbaren Versionen auf.
Im Gegensatz zum \acrshort{bmi} sahen die meisten Verbände und Interessensvertretungen, dass deren Meinungen, Bewertungen und demokratische Teilhabe unerwünscht sei.
Dies liegt insbesondere an den teilweise sehr kurzen Zeiträumen zur Kommentierung der Entwürfe.
So wurde den zum Teil oft ehrenamtlichen Experten lediglich etwas mehr als 24 Stunden (10:15 Uhr bis 14 Uhr des Folgetags) Zeit gegeben,
den 108-seitigen vierten Entwurf zu kommentieren.
Die AG KRITIS bezeichnet die Frist daher auch als den \enquote{ministerielle[n] Mittelfinger ins Gesicht der Zivilgesellschaft}~\cite{it-sig-2.0-vierter-entwurf-24h}.
(vgl.~\cite{it-sig-2.0-alle-vers,it-sig-2.0-vierter-entwurf-24h})

\section{EU-NIS2-Richtlinie}\label{sec:eu-nis2-richtlinie}
An dieser Stelle sei erwähnt, dass die EU 2020 einen ersten Entwurf zur NIS2-Richtlinie, welche die erste ergänzen soll,
erstellt hat.
Auch wenn diese zum Zeitpunkt der Erstellung dieser Seminararbeit noch nicht final ist,
so wurden voraussichtlich einige Punkte bereits mit dem \acrshort{it-sig-2.0} umgesetzt.
Zusätzlich zu den Themen der Informationssicherheit der NIS2-Richtlinie,
reguliert die zugehörige \acrshort{rce}-Richtlinie die Resilienz (Ausfallsicherheit) der KRITIS\@.
(vgl.~\cite{eu-nis-rce})

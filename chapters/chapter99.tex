\chapter{Fazit und Aussicht}\label{ch:fazit-und-aussicht}
Auch, wenn das Gesetz oft kritisiert wurde und einiges im Prozess und Endergebnis nicht zufriedenstellend verlaufen ist,
so war es im Angesicht der fehlenden Sicherheit der informationstechnischen Systeme doch lange überfällig.

Mit der Kritis-Verordnung 1.5, welche am 1.\ Januar 2022 in Kraft getreten ist,
mussten sich neue und alte Anlagen oberhalb der Schwellenwerte bis zum 1.\ April 2022 registrieren
und auch die Maßnahmen nach §~8a umsetzen.
Geprüft werden soll das spätestens zwei Jahre später.
(vgl.~\cite{kritis-verodnun-2021})

Noch dieses Jahr sollen dann auch die letzten erforderlichen Regelungen kommen.
Das wäre zum einen die Kritis-Verodnung 2.0, welche dann auch Angaben zum Sektor Siedlungsabfallentsorgung macht.
Zum Anderen ist das die UBI-VO, welche konkrete Definitionen und Schwellenwerte der \acrshort{unboefi}-Unternehmen liefere.
(vgl.~\cite{kritis-verodnun-2021})

Auch hoffen viele, dass die Evaluierung des \acrshort{it-sig-2.0} im Gegensatz zum \acrshort{it-sig-1.0} diesmal veröffentlicht werde.
Sicherlich wäre es für viele interessant zu erfahren, was das Gesetz letztlich gebracht hat.
Vielleicht wird es aber auch in naher Zukunft ein drittes Änderungsgesetz geben‽
Möglichkeiten zur Verbesserung gibt es schließlich.
Wir dürfen also weiter gespannt sein.

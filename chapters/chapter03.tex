\chapter{KRITIS}\label{ch:kritis}
Nachdem wir nun einen Überblick über die Historie erlangen konnten,
befassen wir uns nun mit den Inhalten des Gesetzes.
In diesem Kapitel werden daher die Änderungen in Bezug auf die Kritische Infrastruktur (KRITIS) beleuchtet.
Während die erste Hälfte des Kapitels sich um die neuen KRITIS Betreiber dreht,
handelt die andere Hälfte von den Pflichten der Betreiber.

\section{Neue KRITIS Betreiber}\label{sec:neue-kritis-betreiber}
Ein wichtiger Teil des \acrshort{it-sig-2.0} dreht sich um das Hinzukommen neuer KRITIS Betreiber.
Hier sind insbesondere zwei große Änderungen zu nennen.

\subsection{Siedlungsabfallentsorgung}\label{ssec:siedlungsabfallentsorgung}
Unter anderem durch die Corona-Pandemie konnten wir feststellen, wie wichtig die Abfallentsorgung eigentlich ist.
Mit Artikel 1 wurde das BSI-Gesetz unter anderem dahingehend geändert,
dass in §~2 Absatz 10 Satz 1 Nummer 1 die Siedlungsabfallentsorgung eingefügt wurde.
Damit ist sie den bisherigen Sektoren Energie, Informationstechnik und Telekommunikation,
Transport und Verkehr, Gesundheit, Wasser, Ernährung, Finanz- und Versicherungswesen nebenan gestellt.
Leider sind zum Zeitpunkt dieser Seminararbeit noch keine genauen Schwellenwerte und Anlagendefinitionen bekannt.
Diese sollen mit der Kritis-Verordnung 2.0 noch im Jahr 2022 geschaffen werden.
(vgl.~\cite{neue-it-sig-2.0})

\subsection{Unternehmen im besonderen öffentlichen Interesse}\label{ssec:unternehmen-im-besonderen-offentlichen-interesse}
Neben den uns meist bekannten Unternehmen der Abfallentsorgung,
fallen auch Unternehmen im besonderen öffentlichen Interesse und deren Zulieferer unter die Regulierungen.
Dabei lassen sich diese den folgenden drei Gruppen zuordnen (§~2 Abs.~14 BSI-Gesetz):
\begin{enumerate}
    \item \textsc{Rüstung}:
    \enquote{Hersteller von Rüstung und Produkten für staatliche Verschlusssachen (VS)}~\cite{neue-it-sig-2.0}
    \item \textsc{Volskswirtschaftliche Bedeutung (und evtl. Zulieferer)}:
    \enquote{Unternehmen von erheblicher volkswirtschaftlicher Bedeutung [...]}~\cite{neue-it-sig-2.0}
    \item \textsc{Gefahrstoffe}:
    \enquote{Betreiber Betriebsbereiche der oberen Klasse mit gefährlichen Stoffen}~\cite{neue-it-sig-2.0}
\end{enumerate}
Auch hier sind die genaueren Definitionen zu diesem Zeitpunkt noch offen und werden ebenfalls voraussichtlich 2022 durch die UBI-VO festgelegt.
Die Pflichten hingegen sind im Allgemeinen bereits in §~8f BSI-Gesetz definiert.
Darunter fallen:
\begin{itemize}
    \item die selbstständige \textsc{Identifikation und Registrierung} innerhalb von zwei Jahren,
    \item regelmäßige \textsc{Selbsterklärungen zur IT-Sicherheit} (Zertifizierungen, Sicherheitsaudits \& Sicherheitsmaßnahmen) vorzunehmen und zu melden und
    \item \textsc{Vorfallsmeldungen}, also bestimmte Störungen, zu melden.
\end{itemize}
(vgl.~\cite{neue-it-sig-2.0})

\section{Angriffserkennung}\label{sec:angriffserkennung}
Zu den wesentlichen technischen und organisatorischen Sicherheitsvorkehrungen der KRITIS-Betreiber gehören mit dem \acrshort{it-sig-2.0} Systeme zur Angriffserkennung.
Grundsätzlich lassen sich dabei zwei Methoden unterscheiden.
Die erste ist die \textsc{Signaturbasierte Angriffserkennung}.
Hierbei wird der Netzwerk- bzw.\ Datenverkehr anhand bekannter Signaturen (z.B.\ durch Zeichenerkennung) analysiert.
Die andere Methode hingegen ist die \textsc{Anomalieerkennung}.
Dabei werden meist mehrere Daten, wie Protokolle, Sensordaten, usw.\ nach Ausreisern untersucht.
Oft werden dazu auch Systeme des maschinellen Lernens genutzt, um die Masse an Daten überhaupt auswerten zu können.
(vgl.~\cite{neue-it-sig-2.0,intrusion-prevention-system})

Laut Begriffsdefinition aus §~2, Abs.~9b BSI-Gesetz handelt es sich bei den Systemen um technische Werkzeuge
und unterstützende Prozesse.
Insbesondere letzteres lässt die Experten dazu schließen, dass wohl \acrshort{siem} und \acrshort{soc} zur Angriffserkennung gemeint sind.
Eine reine Signaturbasierte Angriffserkennung reiche demnach nicht aus.
(vgl.~\cite{neue-it-sig-2.0})

\acrshort{siem} steht hierbei für \textsc{Security Information and Event Management} und ist ein softwarebasiertes Technologiekonzept.
\enquote{Durch das Sammeln, Korrelieren und Auswerten von Meldungen, Alarmen und Logfiles verschiedener Geräte,
Netzkomponenten, Anwendungen und Security-Systeme in Echtzeit werden Angriffe, außergewöhnliche Muster
oder gefährliche Trends sichtbar.}~\cite{siem}
Oft sind die Lösungen auch als Dienste aus der Cloud verfügbar.
(vgl.~\cite{siem})

Größere Unternehmen mit entsprechendem Fachwissen und Resourcen,
können in eigenen \textsc{Security Opterations Centern} (\acrshort{soc}) sich um die Angriffserkennung kümmern.
Auch hier kommen dann meist \acrshort{siem}-Systeme zum Einsatz.
(vgl.~\cite{soc})

Die Webinarreiche zum IT-Sicherheitsgesetz 2.0 unter \url{https://info.rhebo.com/de/webinarreihe-zum-it-sicherheitsgesetz}
ist hierbei sehr zu empfehlen, falls man tiefere Einblicke in dieses Thema erhalten möchte.
So konnte man hier beispielsweise erfahren, dass das Unternehmen Radar Cyber Security der EnBW zunächst SIEM als Cloud-Dienst liefert,
bis nach ungefähr einem halben Jahr die EnBW in der Lage ist, ihr eigenes \acrshort{soc} zu betreiben.
(vgl.~\cite{webinarreihe-it-sig-2.0-auftakt})

Spätestens ab dem 1.\ Mai 2023 ist der Einsatz dann verpflichtend und muss auch explizit nachgewiesen werden.
(vgl.~\cite{neue-it-sig-2.0})

\section{Meldepflichten}\label{sec:meldepflichten}
Neben der Angriffserkennung wurde mit dem §~8b, Abs.~4a BSI-Gesetz die Pflicht auferlegt,
bei erheblichen Störungen dem BSI auf Nachfrage Informationen zur Verfügung zu stellen (einschließlich personenbezogener Daten).
Zusätzlich sind nach §~9b BSI-Gesetz alle kritischen Komponenten dem BSI zu melden.
Das sind unter anderem alle IT-Produkte, die in der Kritischen Infrastruktur eingesetzt werden, also quasi eine Inventarliste.
(vgl.~\cite{neue-it-sig-2.0})

Durch die Änderung des §~8b, Absatz 3 und 3a BSI-Gesetz sind Betreiber unmittelbar nach Feststellung verpflichtet,
sich beim BSI zu registrieren.
Ebenfalls muss damit eine Kontaktstelle benannt werden, welche jederzeit erreichbar sein muss.
Kommt ein Betreiber seiner Registrierungspflicht nicht nach,
kann auch das \acrshort{bsi} selbst die Registrierung samt Benennung der Kontaktstelle vornehmen.
In diesem Fall ist das \acrshort{bsi} auch berechtigt Aufzeichnungen, Schriftstücke und sonstige Unterlagen zu verlangen,
welche zur Bewertung dinglich sind.
Ebenso wird dann auch die zuständige Aufsichtsbehörde informiert.
Daher sollten alle Unternehmen, sobald diese vorliegen, die Kriterien kennen,
um eine mögliche Registrierung nicht zu versäumen.
(vgl.~\cite{neue-it-sig-2.0})

\section{Kritische Komponenten}\label{sec:kritische-komponenten}
Wie bereits im vorherhigen Abschnitt angedeutet,
müssen KRITIS-Betreiber dem \acrshort{bmi} den Einsatz kritischer Komponenten anzeigen.
Allerdings müssen diese und die kritischen Funktionen, aus denen kritische Komponenten abgeleitet werden können,
auf Grund eines Gesetzes festgelegt werden.
Jedoch ist dies bisher lediglich im Sektor Telekommunikation durch das TKG 2021 geschehen.
FÜr alle weiteren Sektoren muss demnach noch abgewartet werden.
Bis dahin gibt es de jure keine kritischen Komponenten in diesen Sektoren.
(vgl.~\cite{neue-it-sig-2.0})

Sollten dann kritische Komponenten eingesetzt werden,
so sind für diese Garantieerklärungen der Vertrauenswürdigkeit des Herstellers erforderlich.
Genaue Anforderungen an solche Erklärungen ist das \acrshort{bmi} noch schuldig.
(vgl.~\cite{neue-it-sig-2.0})

Im Weiteren ist des dem \acrshort{bmi} möglich, unter bestimmten Voraussetzungen,
den Einsatz solcher Komponenten zu untersagen.
Das ist zum einen die \textsc{Beeinträchtigung der öffentlichen Ordnung und Sicherheit},
also beispielsweise, wenn ein Hersteller von der Regierung kontrolliert wird.
Zum anderen kann der Verwendung untersagt werden,
wenn die Vertrauenswürdigkeit basierend auf der Garantieerklärung des Herstellers,
Sicherheitstests und Schwachstellen und Manipulationen fehlt.
Vor allem der zweite Punkt wird aber auch kritisch gesehen, da dies dazu führen könnte,
dass Schwachstellen eher seltener gemeldet würden.
(vgl.~\cite{neue-it-sig-2.0})

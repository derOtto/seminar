\chapter{Einleitung}\label{ch:einleitung}

\section{Motivation}\label{sec:motivation}
Nicht erst seit der Corona-Pandemie lässt sich ein verstärktes Bewusstsein und Verlangen nach Sicherheit in IT-Systemen erkennen.
Durch die immer mehr voranschreitende Digitalisierung werden auch immer mehr, immer neuere Systeme benötigt.
Gleichzeitig steigt damit auch die Wahrscheinlichkeit von Sicherheitslücken,
welche letztlich zu \glslink{it-sicherheitsvorfall}{IT-Sicherheitsvorfällen} führen können.

Doch damit nicht genug.
Nicht nur Hardware, sondern auch Software muss heutzutage von Firmen extern beschafft werden.
Die Folge: Sicherheit kann nicht mehr nur firmenweit beachtet werden.
Vielmehr können auch Angriffe auf Zulieferer erhebliche Folgen nach sich ziehen.
Das zeigt auch der \gls{solarwinds-hack}, durch welchen nochmal bewusst wurde, wie wichtig es ist,
insbesondere kritische Infrastruktur zu schützen.

Unter anderem aus diesen Gründen wurde das \textsc{Zweite Gesetz zur Erhöhung der Sicherheit informationstechnischer Systeme},
meist kurz IT-Sicherheitsgesetz 2.0 genannt, beschlossen.

%viele entwürfe und kritik
%bewusstsein schaffen
%Patch
%Sicherheitsupdates Smartphones

\section{Zielsetzung}\label{sec:zielsetzung}
Diese Seminararbeit soll einen groben Überblick über die Themen des IT-Sicherheitsgestzes 2.0 liefern.
Die Themen werden dabei jeweils kurz beschrieben und erklärt.
Weiter soll die Arbeit dazu dienen, einen ersten Einblick in das Gesetz zu erhalten,
sodass bei Interesse sich die lesende Person entsprechend weiter informieren kann.

\section{Abgrenzung}\label{sec:abgrenzung}
Als Grundlage dient im Allgemeinen die beschlossene Fassung des Gesetzes,
welche im Bundesgesetzblatt Jahrgang 2021 Teil I Nr. 25 am 27. Mai 2021 veröffentlicht wurde.
Die vielen Entwürfe finden somit hier kaum Beachtung.

\section{Aufbau der Arbeit}\label{sec:aufbau-der-arbeit}
Die Arbeit ist in mehrere Kapitel untergliedert.

Zu Beginn der Arbeit wird das Gesetz knapp in den historischen wie europaweiten Kontext gesetzt.

In den weiteren Kapiteln werden die wichtigsten Inhalte aufbereitet und teilweise erläutert.
Der Aufbau und die Schwerpunkte orientieren sich hierbei an
\href{https://www.openkritis.de/it-sicherheitsgesetz/ausblick-it-sicherheitsgesetz-2-0.html}{\textsc{Das neue IT-Sicherheitsgesetz 2.0}}
der OpenKRITIS\@.

Zum Ende dieser Ausarbeitung wird der aktuelle Stand kurz resümiert und ein Ausblick in die Zukunft gegeben.

%*******************************************************
% Abstract in German
%*******************************************************
% TODO content
\begin{otherlanguage}{ngerman}
	\pdfbookmark[0]{Zusammenfassung}{Zusammenfassung}
	\chapter*{Zusammenfassung}
%	Kurze Zusammenfassung des Inhaltes in deutscher Sprache von ca. einer Seite länge. Dabei sollte vor allem auf die folgenden Punkte eingegangen werden:
%	%
%	\begin{itemize}
%	  \item Motivation: Wieso ist diese Arbeit entstanden? Warum ist das Thema der Arbeit (für die Allgemeinheit) interessant? Dabei sollte die Motivation von der konkreten Aufgabenstellung, z.B. durch eine Firma, weitestgehend abstrahiert werden.
%          \item Inhalt: Was ist Inhalt der Arbeit? Was genau wird in der Arbeit behandelt? Hier sollte kurz auf Methodik und Arbeitsweise eingegangen werden.
%          \item Ergebnisse: Was sind die Ergebnisse der Arbeit? Ein kurzer Überblick über die wichtigsten Ergebnisse als Teaser, um die Arbeit vollständig zu lesen.
%	\end{itemize}
%	\medskip
%
%	\noindent
%	Eine großartige Anleitung von Kent Beck, wie man gute Abstracts schreibt, finden Sie hier:
%	\begin{center}
%                \url{https://plg.uwaterloo.ca/~migod/research/beckOOPSLA.html}
%        \end{center}
	Sicherheit in der Informationstechnologie erfährt seit einigen Jahren ein immer größeres Interesse.
	Ein Grund dafür könnte sein, dass die Häufigkeit und Schwere von Angriffen bzw.\ Hacks nie dagewesene Zustände erreicht.
	Trotzdem zeigt dies auch, wie sehr die Thematik bisher vernachläßigt wurde.
	Seit wenigen Jahren hat das auch die Politik erkannt und ist seitdem am Nachbessern.

	Mit dem \acrlong{it-sig-2.0} werden hierbei insbesondere auch in der kritischen Infrastruktur Regeln geschaffen.
	Als \gls{artikelgesetz} ändert es dabei mehrere Gesetze gleichzeitig.
	Nichtsdestotrotz erfährt das BSI-Gesetz die größten Änderungen.

	Diese Seminararbeit soll daher ein zwar umfassendes und verständliches Bild über das \acrshort{it-sig-2.0} verschaffen,
	jedoch gleichzeitig nicht zu tief beispielsweise in andere Gesetze gehen.
	Nach dem Lesen dieser Ausarbeitung soll es dem / der Lesenden möglich sein,
	die Inhalte im Groben zu verstehen und bei Interesse sich entsprechend weiter informieren zu können.
\end{otherlanguage}

\makeglossaries

\newglossaryentry{artikelgesetz}{
    name={Artikelgesetz},
    plural={Artikelgesetze},
    description={\enquote{Ein Artikelgesetz ist ein Gesetz,
    durch das gleichzeitig mehrere Gesetze erlassen oder geändert werden,
    manchmal auch in unterschiedlichen Rechtsgebieten. [...]
    In einer solchen Vorlage sind die Änderungen der verschiedenen Gesetze als Artikel voneinander getrennt.}~\cite{artikelgesetz}},
}

\newglossaryentry{it-sicherheitsvorfall}{
    name={IT-Sicherheitsvorfall},
    plural={IT-Sicherheitsvorfälle},
    description={\enquote{Als IT-Sicherheitsvorfall wird ein Ereignis bezeichnet, bei dem die Vertraulichkeit, Integrität und
    Verfügbarkeit von Informationen, Geschäftsprozessen, IT-Diensten, IT-Systemen oder IT-Anwendungen
    beeinträchtigt werden und als Folge ein großer Schaden entstehen kann.}~\cite[S.~10]{leitfaden-digitale-ersthelfer}},
}

\newglossaryentry{solarwinds-hack}{
    name={SolarWinds-Hack},
    description={2019/2020 wurden durch einen Hack auf die Netzwerkmanagement-Software Orion der Firma SolarWinds etliche Firmen,
    darunter insbesondere auch US-Behörden, gehackt. (vgl.~\cite{der-spionagefall-des-jahres})},
}

\newglossaryentry{eu-richtlinie}{
    name={EU-Richtlinie},
    plural={EU-Richtlinien},
    description={\enquote{Eine [EU-]Richtlinie ist ein Rechtsakt,
    in dem ein von allen EU-Ländern zu erreichendes Ziel festgelegt wird.
    Es ist jedoch Sache der einzelnen Länder,
    eigene Rechtsvorschriften zur Verwirklichung dieses Ziels zu erlassen.}~\cite{eu-verordnungen-richtlinien-sonst}},
}

\newacronym{it-sig-2.0}{IT-SiG 2.0}{IT-Sicherheitsgesetz 2.0}

\newacronym{it-sig-1.0}{IT-SiG 1.0}{IT-Sicherheitsgesetz 1.0}

\newacronym{bsi}{BSI}{Bundesamt für Sicherheit in der Informationstechnik}

\newacronym{bmi}{BMI}{Bundesministerium des Innern und für Heimat}

\newacronym{rce}{RCE}{Resilience of Critical Entities}

\newacronym{siem}{SIEM}{Security Information and Event Management}

\newacronym{soc}{SOC}{Security Operations Center}

\newacronym{unboefi}{UNBÖFI}{Unternehmen im besonderen öffentlichen Interesse}

\newacronym{ics}{ICS}{Industrial Control Systems}

\newacronym{tkg}{TKG}{Telekommunikationsgesetz}

\newacronym{isp}{ISP}{Internet Service Provider}

\newacronym{bfdi}{BfDI}{Bundesbeauftragte für den Datenschutz und die Informationsfreiheit}

\newacronym{ikt}{IKT}{Informations- und Kommunikationstechnik}

\newacronym{dsgvo}{DS-GVO}{Datenschutz-Grundverordnung}
